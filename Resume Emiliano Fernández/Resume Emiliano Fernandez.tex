\documentclass{resume}

\begin{document}

\fontfamily{ppl}\selectfont
\linespread{0.8}

\noindent
% \begin{tabularx}{\linewidth}{@{}m{0.8\textwidth} m{0.2\textwidth}@{}}
% {
\Large{Emiliano Fernández Cervantes} \newline
    \small{
        \clink{\fontfamily{ppl}\selectfont
            Mail: \href{mailto:fdezemi@outlook.com}{fdezemi@outlook.com}  
            Personal Website: \href{https://emilian.website}{emilian.website}
            {\fontdimen2\font=0.75ex Tel.: +1 310-256-5088}
            \newline
            GitHub: \href{https://github.com/EmilianFC20}{github.com/EmilianFC20}
            LinkedIn: \href{https://www.linkedin.com/in/emiliano-fernandez-cervantes/}{linkedin.com/in/emiliano-fernandez-cervantes/} \newline
            
        }
    }
% }
% & 
% {
%     \hfill
%     \includegraphics[width=2.8cm]{images/gr.png}
% }
% \end{tabularx}

\begin{center}
        \csection{PROFESSIONAL PROFILE}{\small
        \begin{itemize}
            \item {Computer Engineering M.S. student specializing in computer architecture and high-performance networking. Eager to apply project experience in hardware design, automation, and network systems to contribute to the development of scalable infrastructure.}
        \end{itemize}
    }
\begin{tabularx}{\linewidth}{@{}*{2}{X}@{}}
% left side %
{
    
    \csection{EDUCATION}{\small
        \begin{itemize}
            % item 1 %
            \item \frcontent{M.S. in Computer Engineering}{}{University of Southern California}{August 2025 - May 2027}
            \item \frcontent{B.S. in Biomedical Engineering}{}{Tecnológico de Monterrey at Mexico City}{August 2016 - December 2020}
            \item \frcontent{Biomedical Engineering Exchange Program}{University of North Texas}{}{August - December 2019}
        \end{itemize}
    }

    \csection{AWARDS \& RECOGNITIONS}{\small
        \begin{itemize}
            % item 1
            \item \frcontent{Viterbi Endowment Scholarship}{Full-tuition merit-based scholarship.}{}{2025-2027}
            % item 2
            \item \frcontent{Fulbright Grant}{Grant to pursue graduate studies in the United States.}{}{2024}
            % item 3
            \item \frcontent{Graduated with Honors and Top of the Class}{Highest GPA in the B.S. in Biomedical Engineering.}{}{2020}
            % item 4
            \item \frcontent{Academic Distinction Scholarship}{70\% tuition merit-based scholarship.}{}{2016-2020}
        \end{itemize}
    }
    
    \csection{SKILLS}{\small
        \begin{itemize}
            \item \textbf{Programming Languages} \newline
            {\footnotesize Python, C/C++, Verilog, CUDA, Bash, Dart, VBA.}{}{}
            \item \textbf{Software \& Tools} \newline
            {\footnotesize Linux (Debian, Fedora/CentOS), Docker, OpenWRT, Git, PyTorch, Jupyter, Modelsim, Cadence Virtuoso, MATLAB, LaTeX.}{}{}
            \item \textbf{Networking} \newline
            {\footnotesize TCP/IP, DNS, DHCP, LAN/WAN, Firewall Configuration.}{}{}
            \item \textbf{Languages} \newline
            {\footnotesize Spanish (native), English (C1 level).}
        \end{itemize}
    }
}
% end left side %
& 
% right side %
{

    \csection{ENGINEERING PROJECTS}{\small
        \begin{itemize}
            \item \frcontent{ARM-Compatible Smart NIC with GPU}{Developing an ARM-based SmartNIC with GPU acceleration for high-throughput, low-latency network processing \href{https://github.com/USC-HW-Engineers/EE533-DPU}{(Github).}}{}{January 2026 - Present}
            \item \frcontent{Pipelined MAC Unit}{Designed and optimized a full-custom 16-bit pipelined MAC unit in Cadence Virtuoso, completing schematic design, transistor-level simulations, and DRC/LVS layout verification.}{}{November 2025}
            \item \frcontent{Local AI Server}{Created a local AI inference server leveraging an NVIDIA GPU with Ollama models and OpenWebui to accelerate AI workloads. Currently integrating a Discord bot in Python for secure, private document querying.}{}{September 2024 - Present}
            \item \frcontent{Home Lab \& Network}{Deployed a Debian-based home server for NAS, home automation, and DNS filtering using Docker. Designed and implemented a reverse proxy enabling SSL and WebSocket support for secure routing. Managed advanced routing, firewall, and network services using OpenWRT.}{}{December 2023-August 2024}
        \end{itemize}
    }

    \csection{ACADEMIC PUBLICATIONS}{\small
        \begin{itemize}
            % item 1 %
            \item \frcontent{Fall Risk Assessment Research Article}{Fernandez, E. et al. (2023). Recurrence quantification analysis of center of pressure trajectories for balance and fall-risk assessment in young and older adults. \textit{IEEE Transactions on Neural Systems and Rehabilitation Engineering}, 31, 926–935. \href{https://doi.org/10.1109/tnsre.2023.3236454}{https://doi.org/10.1109/tnsre.2023.3236454}}{}{}
        \end{itemize}
    }

    \csection{WORK EXPERIENCE}{\small
        \begin{itemize}
            \item \textbf{Sr. Project Support Coordinator PPD, Thermo Fisher Scientific}
            \vspace{-0.2cm}
            \begin{footnotesize}
            \begin{itemize}
                \item Co-developed and deployed a VBA-based tool to automate access reconciliation across multiple systems, creating company-wide time savings and reducing compliance risks for hundreds of projects.
                \item Led support teams, managed timelines, developed technical training materials, and coordinated with management to ensure process adherence and quality across a large portfolio of client projects.
            \end{itemize}
            \vspace{-0.2cm}
            \textit{May 2021 - Aug 2025} 
            \end{footnotesize}
        \end{itemize}
    }

}
\end{tabularx}
\end{center}
\end{document}